\clearpage
%%%%%%%%%%%%%%%%%%%%%%%%%%%%%%%%%%%%%%%%%%%%%%%%%%%%%%%%%%%%%%%%%%%%%%%%%%%%%%%%%
%%        %%%        %%%        %%%        %%%        %%%         %%%  %%%%  %%%
%%  %%%%%%%%%  %%%%%%%%%  %%%%%%%%%%%%  %%%%%%%%%  %%%%%%  %%%%%  %%%    %%  %%%
%%        %%%        %%%  %%%%%%%%%%%%  %%%%%%%%%  %%%%%%  %%%%%  %%%  %  %  %%%
%%%%%%%%  %%%  %%%%%%%%%  %%%%%%%%%%%%  %%%%%%%%%  %%%%%%  %%%%%  %%%  %%    %%%
%%        %%%        %%%        %%%%%%  %%%%%%        %%%         %%%  %%%   %%%
%%%%%%%%%%%%%%%%%%%%%%%%%%%%%%%%%%%%%%%%%%%%%%%%%%%%%%%%%%%%%%%%%%%%%%%%%%%%%%%%%
 \section{Fileからの入力}
 先程出力したファイルからデータを読み込んでヒストグラムを描くことを経験しよう。

 \begin{itembox}{\texttt{filein.cpp}}
\begin{verbatim}
	#include "TCanvas.h"
	#include "TF1.h"
	#include "TH1.h"
	#include "TMath.h"
	#include "TStyle.h"
	#include <fstream>
	TCanvas *filein(char *file_name){
	  double range_min = 0. ;
	  double range_max = 8000.e-9 ;// 8000 [ns]
	  int nbin = 100 ;  
	  double start ; // TDC start
	  double stop  ; // TDC stop
	  double tdc   ; // delta T

	  TCanvas *c1 = new TCanvas("c1","c1",600,600) ;
	  c1->SetGrid(1,1); // Canvas c1 にグリッドを描く
	  c1->SetLogy(1) ;  // Canvas c1 の縦軸をlogで
	  gStyle->SetOptFit(1) ;
	  TH1D *h = new TH1D(file_name,"Decay curve (Muon);TDC [s] ; Counts ",
	  nbin, range_min, range_max);
	  ifstream fin(file_name) ;
	  while(fin >> start >> stop){
	    tdc = stop - start ;
	    h -> Fill(tdc) ;
	  }
	  h->Draw("HE");

	  /******************************************************************************** \
	  * Decay Curve Fitting
	  \********************************************************************************/
	  TF1 *muon = new TF1("muon","[0]*(TMath::Exp(-x/[1])+[2])") ;
	  muon->SetParameters(2e+3, 2e-6, 0.5) ;
	  muon->SetLineColor(kBlue) ;
	  muon->SetLineWidth(4) ;
	  h->Fit(muon) ;

	  return c1 ;
	}
\end{verbatim}
 \end{itembox}