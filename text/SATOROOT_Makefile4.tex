%%%%%%%%%%%%%%%%%%%%%%%%%%%%%%%%%%%%%%%%%%%%%%%%%%%%%%%%%%%%%%%%%%%%%%%%%%%%%%%%%
%%        %%%        %%%        %%%        %%%        %%%         %%%  %%%%  %%%
%%  %%%%%%%%%  %%%%%%%%%  %%%%%%%%%%%%  %%%%%%%%%  %%%%%%  %%%%%  %%%    %%  %%%
%%        %%%        %%%  %%%%%%%%%%%%  %%%%%%%%%  %%%%%%  %%%%%  %%%  %  %  %%%
%%%%%%%%  %%%  %%%%%%%%%  %%%%%%%%%%%%  %%%%%%%%%  %%%%%%  %%%%%  %%%  %%    %%%
%%        %%%        %%%        %%%%%%  %%%%%%        %%%         %%%  %%%   %%%
%%%%%%%%%%%%%%%%%%%%%%%%%%%%%%%%%%%%%%%%%%%%%%%%%%%%%%%%%%%%%%%%%%%%%%%%%%%%%%%%%
 \section{Makefileその4:\ROOT のネイティブプログラミング}
 
 
 
 本章で\ROOT のライブラリなどネイティブでも使用できるようにする。
 
  \subsection{\ROOT ライブラリとのリンク}
  
  \ROOT の数式ライブラリに含まれている定義値$\pi$を表示しよう。
  \begin{itembox}{\texttt{pi.cpp}}
\begin{verbatim}
#include "TMath.h"
#include <iostream>
int main(int argc, char** argv){
  std::cout << "Pi = " << TMath::Pi() << std::endl ;
  return EXIT_SUCCESS ;
}
\end{verbatim}
  \end{itembox}
  メイクファイルを今までと同じように準備してみよう。
  \begin{itembox}{\texttt{pi.cpp}}
\begin{verbatim}
TARGET = pi

COM    = c++
CFLAGS = -Wall
OBJS   = $(TARGET).o

all:$(TARGET)

$(TARGET):$(OBJS)
        $(COM) $(CFLAGS) -o $(TARGET) $(OBJS)
.cpp.o:
        $(CC) $(CFLAGS) -c $<
$(TARGET).o :

clean:
        rm -f *.o
\end{verbatim}
  \end{itembox}
  
  
  \verb|make|を実行してみると、
\begin{verbatim}
$ make
cc -Wall -c pi.cpp
pi.cpp:1:10: fatal error: 'TMath.h' file not found
#include "TMath.h"
         ^
1 error generated.
make: *** [pi.o] Error 1
\end{verbatim}
はい、エラーが出ました。
そこで\texttt{Makefile}を次のように書き換える。
\begin{itembox}{\texttt{Makefile2}}
 \begin{verbatim}
TARGET = pi

COM        = c++
CFLAGS     = -Wall
ROOTCFLAGS = $(shell root-config --cflags)
ROOTLIBS   = $(shell root-config --libs)
CXXFLAGS   = $(ROOTCFLAGS) $(CFLAGS)
CXXLIBS    = $(ROOTLIBS)

OBJS   = $(TARGET).o

all:$(TARGET)

$(TARGET):$(OBJS)
	$(COM) $(CXXLIBS) -o $(TARGET) $(OBJS)
.cpp.o:
	$(CC) $(CXXFLAGS) -c $<
$(TARGET).o :

clean:
	rm -f *.o
 \end{verbatim}
\end{itembox}
行った変更はコンパイルフラグを追加したことと、
\ROOT のライブラリとのリンク作業である。
\verb|make|してみよう。
\begin{verbatim}
$ make -f Makefile2
cc -pthread -m64 -I/usr/local/hep/root/v5.34.09/include/root -Wall -c pi.cpp
c++ -L/usr/local/hep/root/v5.34.09/lib/root -lCore -lCint -lRIO -lNet -lHist \
-lGraf -lGraf3d -lGpad -lTree -lRint -lPostscript -lMatrix -lPhysics -lMathCore \
-lThread -lpthread -lm -ldl -o pi pi.o
$ ./pi 
Pi = 3.14159
\end{verbatim}
ちょっと長ったらしい実行分の後、
目的の実行ファイル\texttt{pi}が出来上がる。


\subsection{幾つかのサンプルプログラム}


   \subsubsection{ヒストグラムの描写と保存}
     \begin{itembox}{\texttt{hist2.cpp}}
\begin{verbatim}
#include "TCanvas.h"
#include "TH1.h"
#include <iostream>

int main(int argc, char** argv){
  TCanvas *c1 = new TCanvas("c1", "c1", 600, 600) ;
  TH1D *h = new TH1D("h", "h", 100, -5., 5.) ;
  h->FillRandom("gaus") ;
  h->Draw("E") ;
  c1->SaveAs("hist2c1.eps") ;
  delete h ;
  delete c1 ;
  return EXIT_SUCCESS ;
}
\end{verbatim}
     \end{itembox}

     \begin{itembox}{\texttt{Makefile}}
\begin{verbatim}
TARGET = hist2

COM        = c++
CFLAGS     = -Wall
ROOTCFLAGS = $(shell root-config --cflags)
ROOTLIBS   = $(shell root-config --libs)
CXXFLAGS   = $(ROOTCFLAGS) $(CFLAGS)
CXXLIBS    = $(ROOTLIBS)

OBJS   = $(TARGET).o

all:$(TARGET)

$(TARGET):$(OBJS)
        $(COM) $(CXXLIBS) -o $(TARGET) $(OBJS)
.cpp.o:
        $(CC) $(CXXFLAGS) -c $<
$(TARGET).o :

clean:
        rm -f *.o
\end{verbatim}
     \end{itembox}
     実行方法は下記の通り。
\begin{verbatim}
$ ls
Makefile	hist2.cpp
$ make
cc -pthread -m64 -I/usr/local/hep/root/v5.34.09/include/root -Wall -c hist2.cpp
c++ -L/usr/local/hep/root/v5.34.09/lib/root -lCore -lCint -lRIO -lNet -lHist -lGraf \
-lGraf3d -lGpad -lTree -lRint -lPostscript -lMatrix -lPhysics -lMathCore -lThread -lpthread \
 -lm -ldl -o hist2 hist2.o
$ ./hist2 
Info in <TCanvas::Print>: eps file hist2c1.eps has been created
$ ls
Makefile	hist2c1.eps		hist2		hist2.cpp	hist2.o
\end{verbatim}

     
   \subsubsection{}

     \begin{itembox}{\texttt{}}
\begin{verbatim}
\end{verbatim}
     \end{itembox}
     \begin{itembox}{\texttt{}}
\begin{verbatim}
\end{verbatim}
     \end{itembox}
   
   
   \subsubsection{}
   
        \begin{itembox}{\texttt{}}
\begin{verbatim}
\end{verbatim}
     \end{itembox}
     \begin{itembox}{\texttt{}}
\begin{verbatim}
\end{verbatim}
     \end{itembox}
