  %%%%%%%%%%%%%%%%%%%%%%%%%%%%%%%%%%%%%%%%%%%%%%%%%%%%%%%%%%%%%%%%%%%%%%%%%%%%%%%%%
  %%        %%%        %%%        %%%        %%%        %%%         %%%  %%%%  %%%
  %%  %%%%%%%%%  %%%%%%%%%  %%%%%%%%%%%%  %%%%%%%%%  %%%%%%  %%%%%  %%%    %%  %%%
  %%        %%%        %%%  %%%%%%%%%%%%  %%%%%%%%%  %%%%%%  %%%%%  %%%  %  %  %%%
  %%%%%%%%  %%%  %%%%%%%%%  %%%%%%%%%%%%  %%%%%%%%%  %%%%%%  %%%%%  %%%  %%    %%%
  %%        %%%        %%%        %%%%%%  %%%%%%        %%%         %%%  %%%   %%%
  %%%%%%%%%%%%%%%%%%%%%%%%%%%%%%%%%%%%%%%%%%%%%%%%%%%%%%%%%%%%%%%%%%%%%%%%%%%%%%%%%
 \section{Makefileその3:引数を利用する}
 
 
 ファイルを実行する時に引数を与えたくなるだろう。
 その簡単な方法を示す。
 いままでおまじないとして
 \verb|main|の引数に\verb|int|型の\verb|argc|、\verb|char**|型の\verb|argv|を記入していた。
 \begin{itemize}
  \item \verb|argc| \\
	実行ファイル$+$与えた引数の数を表す。 \\
	引数が0ならば\verb|argc=1|、
	引数が1つならば\verb|argc=2|、
	引数が2つならば\verb|argc=3|、
	
  \item \verb|argv| \\
	与えた引数を文字列として格納している。\\
	\verb|argv[0]|は実行ファイル名、
	\verb|argv[1]|は第1引数、
	\verb|argv[2]|は第2引数といった具合である。
 \end{itemize}
 
 
 
\subsection{とりあえず動かす}

