  %%%%%%%%%%%%%%%%%%%%%%%%%%%%%%%%%%%%%%%%%%%%%%%%%%%%%%%%%%%%%%%%%%%%%%%%%%%%%%%%%
  %%        %%%        %%%        %%%        %%%        %%%         %%%  %%%%  %%%
  %%  %%%%%%%%%  %%%%%%%%%  %%%%%%%%%%%%  %%%%%%%%%  %%%%%%  %%%%%  %%%    %%  %%%
  %%        %%%        %%%  %%%%%%%%%%%%  %%%%%%%%%  %%%%%%  %%%%%  %%%  %  %  %%%
  %%%%%%%%  %%%  %%%%%%%%%  %%%%%%%%%%%%  %%%%%%%%%  %%%%%%  %%%%%  %%%  %%    %%%
  %%        %%%        %%%        %%%%%%  %%%%%%        %%%         %%%  %%%   %%%
  %%%%%%%%%%%%%%%%%%%%%%%%%%%%%%%%%%%%%%%%%%%%%%%%%%%%%%%%%%%%%%%%%%%%%%%%%%%%%%%%%
 \section{Makefileその1}
 
 

 \subsection{Make(メイク)}

 メイクとはMakefileと呼ばれるルールを記述したファイル、
 または予め定義された規則などからUNIXのShellが実行するコマンド列を作成するコマンドジェネレータの役目を持つ。
 メイクはファイル間の依存関係を整理するのにも使われ、大規模プログラミングでは必須のツールである。
 \verb|make|というコマンドがまず自分のPCに導入されているかを確認しておいて欲しい。
 なければどこぞからインストールしてくること。
 \verb|make|があるかどうかの確認は
 \begin{verbatim}
	$ which make
	/usr/bin/make
 \end{verbatim}
\verb|make|のヴァージョン確認は
 \begin{verbatim}
	$ make -v
GNU Make 3.81
Copyright (C) 2006  Free Software Foundation, Inc.
This is free software; see the source for copying conditions.
There is NO warranty; not even for MERCHANTABILITY or FITNESS FOR A
PARTICULAR PURPOSE.

This program built for i386-apple-darwin11.3.0
 \end{verbatim}
 
 
 
 \subsection{Makefile}
 Makefileメイクで実行するコマンド列を定義しておくファイル名である。
 Makefileが存在するディレクトリ下において、
 \begin{verbatim}
	$ make
 \end{verbatim}
 と入力することでMakefileにかかれた内容が実行される。
 なお、
 \verb|make|で実行したいファイル名が\verb|Makefile2|の場合には
 \begin{verbatim}
	$ make -f Makefile2
 \end{verbatim}
 \verb|make|で実行したいファイル名が\verb|hogex|の場合には
 \begin{verbatim}
	$ make -f hogex
 \end{verbatim}
 などとすればよい。
 
 
 \clearpage
 \subsection{直打ちメイクファイル}
まずは\texttt{hello2.cpp}を\verb|make|することで実行ファイルを作成するMakefileを作ってみよう。
お作法を知らなくてもまずは真似をして書いてみて欲しい。
\begin{itembox}{\texttt{Makefile}}
\begin{verbatim}
# Makefileの中では #で始まる行は実際にはコメントアウトされます。
# hello2.cppからhello2という実行ファイルを作成する為のMakefile
all:hello2

hello2:
        c++ -Wall hello2.cpp -o hello2
\end{verbatim}
\end{itembox}
ただし、\textcolor{red}{'\texttt{c++ -Wall hello2.cpp -o hello2}'の行の行頭は\texttt{<TAB>}を用いること。}

\texttt{hello2.cpp}と\texttt{Makefile}が同じディレクトリにいることを確認した後、
早速\verb|make|を実行し、
実行ファイル\texttt{hello2}が出来上がっていることを確認しよう。
\begin{verbatim}
	$ ls
Makefile	hello2.cpp
	$ make
c++     hello2.cpp   -o hello2
	$ ls
Makefile	hello2		hello2.cpp
\end{verbatim}


\subsubsection*{直打ちメイクファイルの解説}
\texttt{Makefile}の中身を順序立てて追っていく。
\begin{enumerate}
 \item \verb|all:hello2| \\
       コロン記号\verb|:|の直前の\verb|all|のことをターゲットという。
       \verb|make|を実行すると、デフォルトで\verb|all|というターゲットが存在するかどうかをチェックする。
       そして今は\verb|all|が定義されているので、
       この行が処理の対象となる。
       次にターゲットの右にリストされるターゲットから実行する。
       今の場合、\verb|hello2|がリストされているので、
       次のターゲットが\verb|hello2|ということになる。

 \item \verb|hello2:| \\
       ターゲットが\verb|hello2|を意味している。
       この行はターゲットの場所を顕に示しただけであり、
       すぐ次の文へと移る。
       
\item \verb|        c++ -Wall hello2.cpp -o hello2| \\
       ターゲットが\verb|hello2|に属している実行分である。
      \textcolor{red}{Makefileでは実行分の行頭を\texttt{<TAB>}でインデントする必要があるので注意すること。}
      この文で実行される内容は、\verb|hello2.cpp|というソースコードをコンパイルして、
      実行ファイル\verb|hello2|を作るという動作である。
\end{enumerate}



\clearpage
 \subsection{マクロを利用したメイクファイル}
 直打ちメイクファイルはかっこ悪いので少しずつランクアップをしよう。
 \texttt{Makefile}ではマクロという変数を用いることで、
 直打ちから脱却が出来る。
 その例が\texttt{Makefike2}である。
\begin{itembox}{\texttt{Makefile2}}
 \begin{verbatim}
	# hello2.cppからhello2という実行ファイルを作成する為のMakefile
TARGET = hello2
COM    = c++
CFLAGS = -Wall
all:$(TARGET)

$(TARGET):
        $(COM) $(CFLAGS) $(TARGET).cpp -o $(TARGET)
 \end{verbatim}
\end{itembox}
\texttt{マクロ = 値}でマクロを定義してマクロに値を格納し、
\texttt{$(マクロ名)$}でマクロに格納された値を用いる。

\verb|make|を実行し、
実行ファイル\texttt{hello2}が出来上がっていることを確認しよう。
\begin{verbatim}
$ ls
Makefile	Makefile2	hello2.cpp
$ make -f Makefile2
c++ -Wall hello2.cpp -o hello2
$ ls
Makefile	Makefile2	hello2		hello2.cpp
\end{verbatim}
\verb|make|の実行文が全てマクロが展開された形で行われていることが見てとれる。


\subsection{オブジェクトファイルを意識したメイクファイル}

後々の事を考えてオブジェクトファイルを意識した\texttt{Makefile}の例を示す。
\begin{itembox}{\texttt{Makefile3}}
\begin{verbatim}
# hello2.cppからhello2という実行ファイルを作成する為のMakefile
TARGET = hello2
COM    = c++
CFLAGS = -Wall
OBJS   = $(TARGET).o

all:$(TARGET)

$(TARGET):$(OBJS)
        $(COM) $(CFLAGS) $(TARGET).cpp -o $(TARGET)
$(OBJS):$(TARGET).cpp
        $(COM) -c $(TARGET).cpp
\end{verbatim}
\end{itembox}

\subsection{内部マクロを用いたメイクファイル}

\verb|Makefile|内部では、
内部マクロと言われる特殊な指定子を用いることが出来る。

\begin{table}[H]
   \begin{tabular}{ll} \hline
    記号        & 説明 \\ \midrule
    \verb| $@ | &	: ターゲットファイル名 \\
    \verb| $< | &	: 最初の依存ファイル名 \\
    \verb| $? | &	: ターゲットより新しい全ての依存ファイル名 \\
    \verb| $^ | &	: 全ての依存ファイル名 \\
    \verb| $+ | &	: Makefileと同じ順番の依存ファイル名 \\
    \verb| $* | &	: suffixを除いたターゲット名 \\
    \verb| $% | &	: アーカイブだった時のターゲットメンバ名 \\ 
	\hline
  \end{tabular}
\end{table}

内部マクロを用いた\texttt{Makefile}の例を示す。
\begin{itembox}{\texttt{Maefile4}}
 \begin{verbatim}
# hello2.cppからhello2という実行ファイルを作成する為のMakefile
TARGET = hello2
COM    = c++
CFLAGS = -Wall
OBJS   = $(TARGET).o

all:$(TARGET)

$(TARGET):$(OBJS)
        $(COM) $(CFLAGS) $(TARGET).cpp -o $^
$(OBJS):$(TARGET).cpp
        $(COM) -c $<
 \end{verbatim}
\end{itembox}




\subsection{ターゲット\texttt{clean}}

\verb|Makefile|でよく使われるターゲットに\texttt{clean}というのがある。
\texttt{clean}の役割はオブジェクトファイルを削除して全てのファイルを再コンパイル可能にする為のもの。
\begin{itembox}{\texttt{Maefile5}}
 \begin{verbatim}
# hello2.cppからhello2という実行ファイルを作成する為のMakefile
TARGET = hello2
COM    = c++
CFLAGS = -Wall
OBJS   = $(TARGET).o

all:$(TARGET)
$(TARGET):$(OBJS)
        $(COM) $(CFLAGS) $(TARGET).cpp -o $^
$(OBJS):$(TARGET).cpp
        $(COM) -c $<
clean:
        rm -f *.o \end{verbatim}
\end{itembox}
\texttt{Makefile4}と\texttt{Makefile5}の違いは最後の二行だけである。
使い方は
\begin{verbatim}
	$ make -f Makefile5 clean
rm -f *.o
\end{verbatim}