\clearpage
\setcounter{section}{-1} % 章数をゼロから始めます。
%%%%%%%%%%%%%%%%%%%%%%%%%%%%%%%%%%%%%%%%%%%%%%%%%%%%%%%%%%%%%%%%%%%%%%%%%%%%%%%%%
%%        %%%        %%%        %%%        %%%        %%%         %%%  %%%%  %%%
%%  %%%%%%%%%  %%%%%%%%%  %%%%%%%%%%%%  %%%%%%%%%  %%%%%%  %%%%%  %%%    %%  %%%
%%        %%%        %%%  %%%%%%%%%%%%  %%%%%%%%%  %%%%%%  %%%%%  %%%  %  %  %%%
%%%%%%%%  %%%  %%%%%%%%%  %%%%%%%%%%%%  %%%%%%%%%  %%%%%%  %%%%%  %%%  %%    %%%
%%        %%%        %%%        %%%%%%  %%%%%%        %%%         %%%  %%%   %%%
%%%%%%%%%%%%%%%%%%%%%%%%%%%%%%%%%%%%%%%%%%%%%%%%%%%%%%%%%%%%%%%%%%%%%%%%%%%%%%%%%
 \section{前書き}


  \subsection{どんな人向け}
  \begin{itemize}
   \item 実験系の研究室に配属されてデータの解析をする段階になった人
   \item 先輩に「 \ROOT 使えるようになっといてね」とか言われちゃった人
  \end{itemize}
  そんな人の為の覚え書き。
  SATOROOTとは、本ドキュメントの前身を私の後輩が作業する為に作成していたディレクトリ名から拝借した名前である。


  \subsection{方針}
  とりあえず動かす。
  \Cpp の細かいお作法とか正しい言葉の使い方とかは無視。
  動かす上で必要なお作法やおまじないについてはその都度紹介したりしなかったりする。
  とにかく動かせるようにすることを目指す。
  ただし、自分で調べることにも重きを置くのでサンプルを示したらその都度サンプルをいじる練習問題を提供する。


  \subsection{作業環境}
  著者の作業環境は
  \begin{itemize}
   \item 
	 \verb|OS X 10.8.5|
   \item 
	 \verb|ROOT version 5.34/09|
  \end{itemize}


  \subsection{お約束事}
  \begin{itemize}
   \item \verb|$| \ --- \ プロンプトを表す記号。パソコンがユーザーの入力を受け入れる状態を表す。
   \item \verb|root[i]| \ --- \ \verb|i|には数字が入る。
	 コマンドライン上で\ROOT を作業している時の行番号である。
	 \verb|i|を省略することもある。
   \item \verb|SATOROOT| \ --- \ この覚え書きで使用する全てのファイルは\verb|SATOROOT|以下のディレクトリで行う。
  \end{itemize}