  \clearpage
  %%%%%%%%%%%%%%%%%%%%%%%%%%%%%%%%%%%%%%%%%%%%%%%%%%%%%%%%%%%%%%%%%%%%%%%%%%%%%%%%%
  %%        %%%        %%%        %%%        %%%        %%%         %%%  %%%%  %%%
  %%  %%%%%%%%%  %%%%%%%%%  %%%%%%%%%%%%  %%%%%%%%%  %%%%%%  %%%%%  %%%    %%  %%%
  %%        %%%        %%%  %%%%%%%%%%%%  %%%%%%%%%  %%%%%%  %%%%%  %%%  %  %  %%%
  %%%%%%%%  %%%  %%%%%%%%%  %%%%%%%%%%%%  %%%%%%%%%  %%%%%%  %%%%%  %%%  %%    %%%
  %%        %%%        %%%        %%%%%%  %%%%%%        %%%         %%%  %%%   %%%
  %%%%%%%%%%%%%%%%%%%%%%%%%%%%%%%%%%%%%%%%%%%%%%%%%%%%%%%%%%%%%%%%%%%%%%%%%%%%%%%%%
 \section{\ROOT}


  \subsection{\ROOT とは}
  \ROOT ( \url{http://root.cern.ch/drupal/} ) とは、
  高エネルギー業界で広く普及している膨大なデータを効率的に扱うためのフレームワークです。
  \Cpp のお作法でプログラミングします。
  コマントライン上で\ROOT と対話的にプロットやプログラミングを行うことが出来ます。


  \subsection{なぜ\ROOT }
  世の中のいろんなニーズに応えた結果です。(投げやり)


  \subsection{\ROOT のインストール}
  \ROOT のインストール作業を行う。
  \begin{itemize}
   \item \texttt{/usr/local/hep/root/5.34.09} \ --- \ \ROOT のライブラリ置き場
   \item \texttt{~/tmp} \ --- \ コンパイルを実行する時の場所
  \end{itemize}


   \subsubsection*{各ディレクトリの作成}
\begin{verbatim}
	$ sudo mkdir -p /usr/local/hep/root/v5.34.09
	$ mkdir ~/tmp
\end{verbatim}


   \subsubsection*{\ROOT のソースコードのダウンロードと展開}
\begin{verbatim}
	$ cd ~/tmp
	$ sudo wget ftp://root.cern.ch/root/root_v5.34.09.source.tar.gz
	$ ls
	root_v5.34.09.source.tar.gz
	$ sudo tar zxvf root_v5.34.09.source.tar.gz
	$ ls
	root
	root_v5.34.09.source.tar.gz
\end{verbatim}


   \subsubsection*{環境変数の定義}
\begin{verbatim}
	$ export ROOTSYS=/usr/local/hep/root/v5.34.09
\end{verbatim}


   \subsubsection*{インストール作業}
\begin{verbatim}
	$ cd root
	$ sudo ./configure --prefix=/usr/local/hep/root/v5.34.09
\end{verbatim}
以下のコメントが出てくると\texttt{configure}は成功
\begin{verbatim}
	To build ROOT type:

	make
	make  install
\end{verbatim}
指示に従い、
\verb|make|及び\verb|make install|を行う。
コンパイルする。
\begin{verbatim}
	$ make
\end{verbatim}
以下のコメントが出てくると\texttt{make}は成功
\begin{verbatim}
============================================================
===                ROOT BUILD SUCCESSFUL.                ===
=== Run 'make install' now.                              ===
============================================================
\end{verbatim}
インストールする。
\begin{verbatim}
	$ su
Password:
	$ make install
	...
	$ exit
\end{verbatim}


\subsubsection*{一時ファイルの削除}
\begin{verbatim}
	$ cd ../
	$ pwd
	~/tmp
	$ rm -rf root
\end{verbatim}


\subsubsection*{環境変数ファイルの作成}
\ROOT 用の環境変数定義を書き込んだ\verb|setup.sh|を準備して、
ホームディレクトリに置く。
\begin{itembox}{\texttt{setup.sh}}
\begin{verbatim}
export ROOTSYS=/usr/local/hep/root/v5.34.09
export PATH=${ROOTSYS}/bin:${PATH} 
export LD_LIBRARY_PATH=${ROOTSYS}/lib/root:${LD_LIBRARY_PATH}
\end{verbatim}
\end{itembox}
ホームディレクトリ内のファイル\verb|.bash_profile|に以下の一文を追加する。
\begin{verbatim}
	source /usr/local/hep/root/setup.sh
\end{verbatim}
その後、
\begin{verbatim}
	source .bash_profile
\end{verbatim}
