\clearpage
%%%%%%%%%%%%%%%%%%%%%%%%%%%%%%%%%%%%%%%%%%%%%%%%%%%%%%%%%%%%%%%%%%%%%%%%%%%%%%%%%
%%        %%%        %%%        %%%        %%%        %%%         %%%  %%%%  %%%
%%  %%%%%%%%%  %%%%%%%%%  %%%%%%%%%%%%  %%%%%%%%%  %%%%%%  %%%%%  %%%    %%  %%%
%%        %%%        %%%  %%%%%%%%%%%%  %%%%%%%%%  %%%%%%  %%%%%  %%%  %  %  %%%
%%%%%%%%  %%%  %%%%%%%%%  %%%%%%%%%%%%  %%%%%%%%%  %%%%%%  %%%%%  %%%  %%    %%%
%%        %%%        %%%        %%%%%%  %%%%%%        %%%         %%%  %%%   %%%
%%%%%%%%%%%%%%%%%%%%%%%%%%%%%%%%%%%%%%%%%%%%%%%%%%%%%%%%%%%%%%%%%%%%%%%%%%%%%%%%%
 \section{Fileへの出力}

 \begin{itembox}{\texttt{fileout.cpp}}
\begin{verbatim}
	#include "TCanvas.h"
	#include "TH1.h"
	#include "TMath.h"
	#include "TRandom3.h"
	#include <fstream>
	#include <iostream>
	TCanvas *fileout(){
	  std::ofstream outputfile; // file の出力先
	  outputfile.open("output.plt" );// file を開く

	  TRandom3 Random(unsigned(time(NULL)));

	  TCanvas *c1 = new TCanvas("c1","c1") ;

	  int imax = 100000; // event 数の最大値
	  double start ; // start の時刻を擬似的に与える。 
	  double stop  ; // stop の時刻を擬似的に与える。 
	  double tdc   ; // TDC でのカウント数を擬似的に与える。
	  double life = 2.2e-6 ; // Muon の平均寿命を入力する。

	  TH1D *h1 = new TH1D("h1","h1",200,0,8000e-9) ; // [s]
	  for(int i = 0; i < imax; i++){
	    start = Random.Uniform(0., 1.e-9) ; // start の時刻を擬似的に与える。
	    stop = Random.Exp(life) ; // 指数関数に従った乱数だけ立った時刻を stop とする
	    stop += start ; // start してから 指数関数に従った乱数だけ立った時刻を stop とする
	    tdc = stop - start ; // このプログラムだけを見るとこの処理は余分な処理だが、TDC の理の為
	    std::cout << tdc << " [s] :: " << start << "" << stop << std::endl;
	    // output という出力先に start と stop をスペース区切りで出力する
	    outputfile << start << " " << stop << " " << endl ;
	    h1->Fill(tdc) ; // ヒストグラムに詰める
	  }
	  h1->Draw("H") ; // 確認用にヒストグラムを出力する。
	  outputfile.close(); // ファイルを閉じる return c1 ;
	}
\end{verbatim}
 \end{itembox}

 \begin{itembox}{\texttt{output.plt}}
\begin{verbatim}
	4.8318e-10 1.59013e-06 
	8.88674e-11 2.45466e-06 
	8.04413e-10 1.65805e-06 
	...
\end{verbatim}
 \end{itembox}

  \subsection{練習}
  \begin{enumerate}
   \item プログラムの各行を理解せよ
	 \begin{description}
	  \item[ヒント] \verb|<fstream>| なるものを\verb|include|している。これはファイル操作の時に今回使用するライブラリである。(今回は使用していないが、\ROOT には\verb|TFile.h|なるクラスも存在する。)
	 \end{description}

   %  \item 出力先のファイル名を今はマクロ内に直書きしているが、%出力先のファイルを output2.plt にしたい時に
	 %\begin{verbatim}
	 %	root[0] .L fileoutsol1.cpp+
	 %	root[1] fileout("output2.plt") 
	  %\end{verbatim}
	 %	としたら、自動的に出力先がoutput2.pltに出来る仕様に変更せよ.
	 %	\begin{description}
%	 \item[ヒント]  \verb|TCanvas *fileout(){HogeHoge }|において引数の中身(\verb|()|の中身)は今まで何も書いてきませんでしたが、
%		    この中には引数を入れることができます。引数をint型の数字\verb|_intnum|にしたいのであれば \\ \verb|TCanvas *fileout(int _intnum ){HogeHoge }|などとして、\verb|{}|内では\verb|_intnum|を定義された\verb|int|型数字だと扱えばいいだけである。では文字列ではどうすればよいか。

\item 出力先のファイル名を今はマクロ内に直書きしているが、引数をファイル名にせよ
  \end{enumerate}

\subsection{解答例}
\begin{enumerate}
\item プログラムの各行を理解せよ

\item 出力先のファイル名を今はマクロ内に直書きしているが、引数をファイル名にせよ
      
      \begin{itembox}{\texttt{fileoutsol1.cpp}}
\begin{verbatim}
	...
	TCanvas *fileoutsol1(char *outputfilename){
	  std::ofstream outputfile; // file の出力先
	  outputfile.open(outputfilename );// file を開く
	  ...
	}
\end{verbatim}
      \end{itembox}
      
\begin{verbatim}
root[0] .L fileoutsol1.cpp+
root[1] fileoutsol1("output2.plt") 
\end{verbatim} 

\end{enumerate}