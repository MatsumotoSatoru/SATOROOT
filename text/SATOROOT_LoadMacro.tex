\clearpage
%%%%%%%%%%%%%%%%%%%%%%%%%%%%%%%%%%%%%%%%%%%%%%%%%%%%%%%%%%%%%%%%%%%%%%%%%%%%%%%%%
%%        %%%        %%%        %%%        %%%        %%%         %%%  %%%%  %%%
%%  %%%%%%%%%  %%%%%%%%%  %%%%%%%%%%%%  %%%%%%%%%  %%%%%%  %%%%%  %%%    %%  %%%
%%        %%%        %%%  %%%%%%%%%%%%  %%%%%%%%%  %%%%%%  %%%%%  %%%  %  %  %%%
%%%%%%%%  %%%  %%%%%%%%%  %%%%%%%%%%%%  %%%%%%%%%  %%%%%%  %%%%%  %%%  %%    %%%
%%        %%%        %%%        %%%%%%  %%%%%%        %%%         %%%  %%%   %%%
%%%%%%%%%%%%%%%%%%%%%%%%%%%%%%%%%%%%%%%%%%%%%%%%%%%%%%%%%%%%%%%%%%%%%%%%%%%%%%%%%
 \section{\ROOT でマクロを読む/実行する方法}
 \ROOT への命令文が書かれたプログラムのソースコード\verb|hello.cpp|を準備しよう。
 \begin{itembox}{\texttt{hello.cpp}}
\begin{verbatim}
	#include <iostream>
	void hello(){
	std::cout << "Hello, world" << std::endl ;
	}
\end{verbatim}
 \end{itembox}
 ざっとプログラムの流れを眺める。
 \begin{enumerate}
  \item \verb|#include <iostream>| \\
	\verb|#include|とは、\Cpp のお作法であってヘッダファイル/ライブラリを読み込むということである。
	今の場合\Cpp で標準入出力を行う為に必要となるライブラリ\verb|<iostream>|を読み込むという意味
  \item \verb|void hello(){| \\
	マクロの名前を準備するかを記述している箇所である。
	\Cpp では関数を定義する時に \\
	<型名> <関数名>(<引数>)\{HogeHoge\} \\
	という書き方をする。今の場合だと\\
	<型名> = \verb|void|、<関数名>=\verb|hello|、<引数>=無し、といった具合である。
	\textcolor{red}{暫くの間はマクロ名とファイル名を同じにする。}

  \item \verb|std::cout << "Hello, world" << std::endl ;| \\
	\verb|std::cout|は標準出力を意味している。
	\verb|<<|は標準出力先に続き文字列や数字を渡す。
	\verb|"Hello, world"|を標準出力に渡している。
	\verb|std::endl|は改行を意味する。

  \item \verb|}| \\
	2行目の\verb|{|に対応する括弧
 \end{enumerate}

 今はソースコードを理解しなくてもいいが、
 気になる人は付録\ref{sec:namespace}などを参考にして理解せよ。


  \subsection{\texttt{hello.cpp}の実行方法1}
  \verb|hello.cpp|を\ROOT 起動時に実行するには、
\begin{verbatim}
	$ root hello.cpp
\end{verbatim}
上記のコマンドを実行すると、
\begin{verbatim}
	root [0] 
	Processing hello.cpp...
	Hello, World
\end{verbatim}

  \subsection{\texttt{hello.cpp}の実行方法2}
  別の方法は\ROOT をいったん起動して、プログラムを"ロード"して実行するという手段。
\begin{verbatim}
	$ root
	root [0] .L hello.cpp 
	root [1] hello()
	Hello, World
\end{verbatim}

  \subsection{\texttt{hello.cpp}の実行方法3(推奨)}
  ロードする時に\Cpp コンパイラを通してマクロに含まれるエラーメッセージなどを表記してくれる方法。
  やり方は'\verb|+|'をロードするファイル名の末尾につける。
  ( \url{http://root.cern.ch/drupal/content/compiling-macros} )
\begin{verbatim}
	$ root
	root [0] .L hello.cpp+
\end{verbatim}

例えば\verb|hello.cpp|の\verb|std::endl;|のセミコロンが無い時にはどうなるかというと、
下のようにエラーメッセージと何が悪いのかをコンパイラが返してくれる。
\begin{verbatim}
	Info in <TUnixSystem::ACLiC>: creating shared library /Users/SATOROOT/hello_cpp.so
	In file included from /Users/SATOROOT/hello_cpp_ACLiC_dict.cxx:17:
	In file included from /Users/SATOROOT/hello_cpp_ACLiC_dict.h:34:
	/Users/SATOROOT/hello.cpp:3:42: error: expected ';' after expression
	std::cout << "Hello, World" <<std::endl 
	^
	;
	In file included from /Users/SATOROOT/hello_cpp_ACLiC_dict.cxx:17:
	In file included from /Users/SATOROOT/hello_cpp_ACLiC_dict.h:18:
	/usr/local/hep/root/v5.34.09/include/root/G__ci.h:971:7:  \
	warning: private field 'type' is not used [-Wunused-private-field]
	int type;
	^
	/usr/local/hep/root/v5.34.09/include/root/G__ci.h:972:7: \
	warning: private field 'tagnum' is not used [-Wunused-private-field]
	int tagnum;
	^
	/usr/local/hep/root/v5.34.09/include/root/G__ci.h:973:7: \
	warning: private field 'typenum' is not used [-Wunused-private-field]
	int typenum;
	^
	/usr/local/hep/root/v5.34.09/include/root/G__ci.h:975:19: warning: \
	private field 'isconst' is not used
	[-Wunused-private-field]
	G__SIGNEDCHAR_T isconst;
	^
	/usr/local/hep/root/v5.34.09/include/root/G__ci.h:977:29: warning: \
	private field 'dummyForCint7' is not used
	[-Wunused-private-field]
	struct G__DUMMY_FOR_CINT7 dummyForCint7;
	^
	5 warnings and 1 error generated.
	clang: error: no such file or directory: '/Users/SATOROOT/hello_cpp_ACLiC_dict.o'
	Error in <ACLiC>: Compilation failed!
\end{verbatim}