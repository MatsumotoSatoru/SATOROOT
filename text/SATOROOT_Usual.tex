\clearpage
%%%%%%%%%%%%%%%%%%%%%%%%%%%%%%%%%%%%%%%%%%%%%%%%%%%%%%%%%%%%%%%%%%%%%%%%%%%%%%%%
%%        %%%        %%%        %%%        %%%        %%%         %%%  %%%%  %%%
%%  %%%%%%%%%  %%%%%%%%%  %%%%%%%%%%%%  %%%%%%%%%  %%%%%%  %%%%%  %%%    %%  %%%
%%        %%%        %%%  %%%%%%%%%%%%  %%%%%%%%%  %%%%%%  %%%%%  %%%  %  %  %%%
%%%%%%%%  %%%  %%%%%%%%%  %%%%%%%%%%%%  %%%%%%%%%  %%%%%%  %%%%%  %%%  %%    %%%
%%        %%%        %%%        %%%%%%  %%%%%%        %%%         %%%  %%%   %%%
%%%%%%%%%%%%%%%%%%%%%%%%%%%%%%%%%%%%%%%%%%%%%%%%%%%%%%%%%%%%%%%%%%%%%%%%%%%%%%%%
 \section{幾つかの常套手段}
 
 
  \subsection{引数を数字にする}
  引数としてある整数を与えて、
  その数自身、その数の二乗、その数の三乗を出力するサンプルプログラムが\verb|usual1.cpp|である。
  \begin{itembox}{\texttt{usual1.cpp}}
\begin{verbatim}
#include <iostream>
#include "TMath.h"
void usual1(int i){
  std::cout << i << std::endl ;
  std::cout << TMath::Power(i,2) << std::endl ;
  std::cout << TMath::Power(i,3) << std::endl ;
}
\end{verbatim}
  \end{itembox}
実行方法は次の通りである。
\begin{verbatim}
$ root
root [0] .L usual1.cpp+
root [1] usual1(3)
3
9
27
\end{verbatim}
なお、\verb|#include "TMath.h"|という命令文によって、
\ROOT に組み込まれている定数や数式演算を使用可能にしている。
\url{http://root.cern.ch/root/html/TMath.html}


  \subsection{引数を文字列にする}
  引数を文字列にしたい場合には
  \begin{itembox}{\texttt{usualchar.cpp}}
\begin{verbatim}
#include <iostream>
void usualchar(char *name){
  std::cout << "character string = " << name  << std::endl ;
}
\end{verbatim}
  \end{itembox}
実行方法は次の通りである。
\begin{verbatim}
$ root
root [0] .L usualchar.cpp 
root [1] usualchar("test")
character string = test
\end{verbatim}


  \subsection{引数を複数にする}
  複数個の引数を与えたい時には引数の\verb|()|の中に、
  \verb|,|を挟んで定義すれば良い。
  \begin{itembox}{\texttt{usualage.cpp}}
\begin{verbatim}
#include <iostream>
void usualage(char *name,int i){
  std::cout << name << " is " << i << " years old." << std::endl ;
}
\end{verbatim}
  \end{itembox}
実行方法は次の通りである。
\begin{verbatim}
$ root
root [0] .L usualage.cpp 
root [1] usualage("satoru",24)
satoru is 24 years old.
\end{verbatim}