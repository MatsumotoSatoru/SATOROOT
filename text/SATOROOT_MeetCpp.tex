以下は作業中。


\texttt{C}言語もしくはその流れを汲む\texttt{C++}を初めて触れる人の為の最低限を紹介する。
 人間が書いたソースコードを機械が理解出来る言語に変換することをコンパイルという。
 \texttt{C++}のソースコードをコンパイルする方法を紹介する。



\section{何もしないプログラムを作成・実行する}


 \subsection{何もしないプログラムを用意する}
 まずは何もしない空っぽのプログラムを書いてみる。
 \begin{itembox}{\texttt{empty.cpp}}
\begin{verbatim}
int main(){
}
\end{verbatim}
 \end{itembox}
\texttt{empty.cpp}は\texttt{C++}で書かれた本当に何もしないプログラムである。


 \subsection{初めてのコンパイル}
 まずは自分の作業環境にコンパイラが存在することを確認する。
 本テキストでは一貫して\verb|c++|を用いる。
 \texttt{empty.cpp}をコンパイルする方法は以下のようである。
\begin{verbatim}
 $ c++ empty.cpp
\end{verbatim}
すると、\texttt{a.out}というファイルがコンパイルしたディレクトリに作成される。
ただし、\texttt{empty.cpp}の内容を正しく入力出来ていない場合にはエラー分が出てくるのでその指示に従ってなおす。
\texttt{a.out}の実行方法は
\begin{verbatim}
 $ ./a.out
\end{verbatim}
とすればよい。
先に述べた通り、何も起きない。


\section{Hello, world}
プログラムに関わる人にはおなじみの\texttt{Hello, world}を出力するプログラムを作成する。
\begin{itembox}{\texttt{hello.cpp}}
\begin{verbatim}
#include <iostream>
int main(){
  std::cout << "Hello, World" << std::endl ;
  retrun 0 ;
}
\end{verbatim}
\end{itembox}
各行の簡単な説明を行う。
{\color{red}\begin{verbatim}
#include <iostream>
\end{verbatim} }
\texttt{C}言語はライブラリと呼ばれるものを読み込むことで、
そのライブラリ内で定義されている様々な操作を行うことが可能となる。
この'読み込む'ということが\ \verb|#include|\ であり、
'読み込まれるライブラリ'が\ \texttt{iostream}\ である。
{\color{red}\begin{verbatim}
int main(){
\end{verbatim}}
\texttt{C}言語で書かれたプログラムが実行されたとき、
一番最初に実行する関数の名前は\ \texttt{main}\ という関数である。
この一行はその定義文である。
{\color{red}\begin{verbatim}
std::cout << "Hello, world" << std::endl ;
\end{verbatim}}
行末のセミコロンは\texttt{C}言語では改行を意味している。
\texttt{std::cout}は続く値や文章を標準出力に出力することをあらわす。
なお\texttt{std::}は名前空間という概念(\ref{sec:namespace})が関係するが
後々勉強すればよい。
感覚的には\texttt{std::cout}という標準出力にむけて\texttt{<<}という記号で持って、
文字列\texttt{"Hello, world"}を送り出している。
その後、続けて\texttt{std::endl}を送り出している。
\texttt{std::endl}は改行を意味する。
{\color{red}\begin{verbatim}
return 0 ;
\end{verbatim}}
main関数の実行結果をOSに伝えている一文であり、
0は正常終了を意味する。
0以外は異常終了を意味する。
一般的には\texttt{return 1;}などして明示的に異常終了を伝える。
{\color{red}\begin{verbatim}
}
\end{verbatim}}
main関数に対応する括弧である。

最後にコンパイルして、実行してみよう。
\begin{verbatim}
$ c++ empty.cpp
$ ./a.out
Hello, world
\end{verbatim}

\section{変数の宣言}
  \subsection{変数の型}
\texttt{C}言語には以下の様な型が存在する。
以下はそれぞれどの程度のデータを確保できるかというデータサイズが決まっているが、
使用環境にもよる。
ここでは紹介程度ですますので、
より詳しいことが知りたい人は各自で検索。
\begin{table}[H]
 \begin{tabular}{ll}
  型 & 何に使うか \\
  char   & 文字列 \\
  int    & 整数(小数は扱えません) \\
  float  & 数字 \\
  double & 数字(floatよりも高い精度) \\
 \end{tabular}
 \caption{\texttt{C}言語の型}
\end{table}

  
\section{しつこり Hello, world}
