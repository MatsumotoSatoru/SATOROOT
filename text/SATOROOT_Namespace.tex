 %%%%%%%%%%%%%%%%%%%%%%%%%%%%%%%%%%%%%%%%%%%%%%%%%%%%%%%%%%%%%%%%%%%%%%%%%%%%%%%%%
 %%        %%%        %%%        %%%        %%%        %%%         %%%  %%%%  %%%
 %%  %%%%%%%%%  %%%%%%%%%  %%%%%%%%%%%%  %%%%%%%%%  %%%%%%  %%%%%  %%%    %%  %%%
 %%        %%%        %%%  %%%%%%%%%%%%  %%%%%%%%%  %%%%%%  %%%%%  %%%  %  %  %%%
 %%%%%%%%  %%%  %%%%%%%%%  %%%%%%%%%%%%  %%%%%%%%%  %%%%%%  %%%%%  %%%  %%    %%%
 %%        %%%        %%%        %%%%%%  %%%%%%        %%%         %%%  %%%   %%%
 %%%%%%%%%%%%%%%%%%%%%%%%%%%%%%%%%%%%%%%%%%%%%%%%%%%%%%%%%%%%%%%%%%%%%%%%%%%%%%%%%
 \section{名前空間\label{sec:namespace}}


 「あめ」といった時にそれがどういう内容を表すだろうか?
 識別する方法としては、
 \begin{enumerate}
  \item 「「気象現象」に属する「あめ」」
  \item「「食べ物」に属する「あめ」」
 \end{enumerate}
 などとしてしまえば、
 「あめ」の表す内容は明確になる。
 この時の「気象現象」や「食べ物」のようなくくりに当たる概念が名前空間である。
 この時使用した「属する」という言葉を\Cpp ではスコープ演算子\verb|"::"|で表す。
 つまり、先程の例を\Cpp 風に表現すると下記のようになる。
 \begin{enumerate}
  \item 気象現象\verb|::|あめ
  \item 食べ物\verb|::|あめ
 \end{enumerate}

  \subsection{名前空間\texttt{std::}}
  プログラム中で
\begin{verbatim}
	#include <iostream>
\end{verbatim}
と宣言すれば、
名前空間\texttt{std}が使用可能となる。
具体的な使い方としては、
\begin{verbatim}
	std::cout << "abc" << std::endl;
\end{verbatim}
などである。
意味としては、
\verb|abc|という文字列と\verb|std::endl|という改行命令を\verb|std::cout|で設定されている標準出力外面へ出力する。


  \subsection{名前空間\texttt{TMath::}}
  プログラム中で
\begin{verbatim}
	#include "TMath.h"
\end{verbatim}
と宣言すれば、
名前空間\texttt{TMath}が使用可能となる。
\url{http://root.cern.ch/root/html/TMath.html}
またコマンドライン上で\verb|ROOT|を使用する時には宣言の必要はない。
\begin{verbatim}
	root [] TMath::C()
	(Double_t)2.99792458000000000e+08
	root [] TMath::Pi()
	(Double_t)3.14159265358979312e+00
	root [] TMath::Power(2,3)
	(Double_t)8.00000000000000000e+00
	root [] TMath::Abs(-2.)
	(Double_t)2.00000000000000000e+00
	root [] TMath::Sin(1.)
	(Double_t)8.41470984807896505e-01
\end{verbatim}
などである。
上記の入力や引数や返り値の意味することは各自で調べよ。
